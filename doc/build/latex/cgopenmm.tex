%% Generated by Sphinx.
\def\sphinxdocclass{report}
\documentclass[letterpaper,12pt,english,openany,oneside]{sphinxmanual}
\ifdefined\pdfpxdimen
   \let\sphinxpxdimen\pdfpxdimen\else\newdimen\sphinxpxdimen
\fi \sphinxpxdimen=.75bp\relax

\PassOptionsToPackage{warn}{textcomp}
\usepackage[utf8]{inputenc}
\ifdefined\DeclareUnicodeCharacter
% support both utf8 and utf8x syntaxes
\edef\sphinxdqmaybe{\ifdefined\DeclareUnicodeCharacterAsOptional\string"\fi}
  \DeclareUnicodeCharacter{\sphinxdqmaybe00A0}{\nobreakspace}
  \DeclareUnicodeCharacter{\sphinxdqmaybe2500}{\sphinxunichar{2500}}
  \DeclareUnicodeCharacter{\sphinxdqmaybe2502}{\sphinxunichar{2502}}
  \DeclareUnicodeCharacter{\sphinxdqmaybe2514}{\sphinxunichar{2514}}
  \DeclareUnicodeCharacter{\sphinxdqmaybe251C}{\sphinxunichar{251C}}
  \DeclareUnicodeCharacter{\sphinxdqmaybe2572}{\textbackslash}
\fi
\usepackage{cmap}
\usepackage[T1]{fontenc}
\usepackage{amsmath,amssymb,amstext}
\usepackage{babel}
\usepackage{times}
\usepackage[Bjarne]{fncychap}
\usepackage{sphinx}

\fvset{fontsize=\small}
\usepackage{geometry}

% Include hyperref last.
\usepackage{hyperref}
% Fix anchor placement for figures with captions.
\usepackage{hypcap}% it must be loaded after hyperref.
% Set up styles of URL: it should be placed after hyperref.
\urlstyle{same}

\addto\captionsenglish{\renewcommand{\figurename}{Fig.\@ }}
\makeatletter
\def\fnum@figure{\figurename\thefigure{}}
\makeatother
\addto\captionsenglish{\renewcommand{\tablename}{Table }}
\makeatletter
\def\fnum@table{\tablename\thetable{}}
\makeatother
\addto\captionsenglish{\renewcommand{\literalblockname}{Listing}}

\addto\captionsenglish{\renewcommand{\literalblockcontinuedname}{continued from previous page}}
\addto\captionsenglish{\renewcommand{\literalblockcontinuesname}{continues on next page}}
\addto\captionsenglish{\renewcommand{\sphinxnonalphabeticalgroupname}{Non-alphabetical}}
\addto\captionsenglish{\renewcommand{\sphinxsymbolsname}{Symbols}}
\addto\captionsenglish{\renewcommand{\sphinxnumbersname}{Numbers}}

\addto\extrasenglish{\def\pageautorefname{page}}

\setcounter{tocdepth}{1}



\title{Coarse-grained OpenMM Documentation}
\date{Jun 04, 2019}
\release{0.0.1}
\author{Garrett A. Meek\\Lenny T. Fobe\\ \\Research group of Michael R. Shirts\\ \\Dept. of Chemical and Biological Engineering\\University of Colorado Boulder}
\newcommand{\sphinxlogo}{\vbox{}}
\renewcommand{\releasename}{Release}
\makeindex
\begin{document}

\pagestyle{empty}
\sphinxmaketitle
\pagestyle{plain}
\sphinxtableofcontents
\pagestyle{normal}
\phantomsection\label{\detokenize{index::doc}}


This documentation is generated automatically using Sphinx, which reads all docstring-formatted comments from Python functions in the ‘cg\_openmm’ repository.  (See cg\_openmm/doc for Sphinx source files.)


\chapter{OpenMM Simulation() protocols for coarse grained modeling}
\label{\detokenize{build:openmm-simulation-protocols-for-coarse-grained-modeling}}\label{\detokenize{build::doc}}
This page details the functions in cg\_openmm/src/build/cg\_build.py.

\phantomsection\label{\detokenize{build:module-build.cg_build}}\index{build.cg\_build (module)@\spxentry{build.cg\_build}\spxextra{module}}\index{add\_new\_elements() (in module build.cg\_build)@\spxentry{add\_new\_elements()}\spxextra{in module build.cg\_build}}

\begin{fulllineitems}
\phantomsection\label{\detokenize{build:build.cg_build.add_new_elements}}\pysiglinewithargsret{\sphinxcode{\sphinxupquote{build.cg\_build.}}\sphinxbfcode{\sphinxupquote{add\_new\_elements}}}{\emph{cgmodel}, \emph{list\_of\_masses}}{}
Adds new coarse grained particle types to OpenMM

cgmodel: CGModel() class object

list\_of\_masses: List of masses for the particles we want to add to OpenMM

\end{fulllineitems}

\index{build\_mm\_force() (in module build.cg\_build)@\spxentry{build\_mm\_force()}\spxextra{in module build.cg\_build}}

\begin{fulllineitems}
\phantomsection\label{\detokenize{build:build.cg_build.build_mm_force}}\pysiglinewithargsret{\sphinxcode{\sphinxupquote{build.cg\_build.}}\sphinxbfcode{\sphinxupquote{build\_mm\_force}}}{\emph{sigma}, \emph{epsilon}, \emph{charge}, \emph{num\_beads}, \emph{cutoff=Quantity(value=1}, \emph{unit=nanometer)}}{}
Build an OpenMM ‘Force’ for the non-bonded interactions in our model.

sigma: Non-bonded bead Lennard-Jones interaction distances,
( float * simtk.unit.distance )

epsilon: Non-bonded bead Lennard-Jones interaction strength,
( float * simtk.unit.energy )

charge: Charge for all beads
( float * simtk.unit.charge )

cutoff: Cutoff distance for nonbonded interactions
( float * simtk.unit.distance )

num\_beads: Total number of beads in our coarse grained model
( integer )

\end{fulllineitems}

\index{build\_mm\_simulation() (in module build.cg\_build)@\spxentry{build\_mm\_simulation()}\spxextra{in module build.cg\_build}}

\begin{fulllineitems}
\phantomsection\label{\detokenize{build:build.cg_build.build_mm_simulation}}\pysiglinewithargsret{\sphinxcode{\sphinxupquote{build.cg\_build.}}\sphinxbfcode{\sphinxupquote{build\_mm\_simulation}}}{\emph{topology}, \emph{system}, \emph{positions}, \emph{temperature=Quantity(value=300.0}, \emph{unit=kelvin)}, \emph{simulation\_time\_step=None}, \emph{total\_simulation\_time=Quantity(value=1.0}, \emph{unit=picosecond)}, \emph{output\_pdb='output.pdb'}, \emph{output\_data='output.dat'}, \emph{print\_frequency=100}}{}
Construct an OpenMM simulation object for our coarse grained model.

topology: OpenMM topology object

system: OpenMM system object

positions: Array containing the positions of all beads
in the coarse grained model
( np.array( ‘num\_beads’ x 3 , ( float * simtk.unit.distance ) )

temperature: Simulation temperature ( float * simtk.unit.temperature )

simulation\_time\_step: Simulation integration time step
( float * simtk.unit.time )

total\_simulation\_time: Total simulation time ( float * simtk.unit.time )

output\_data: Name of output file where we will write the data from this
simulation ( string )

print\_frequency: Number of simulation steps to skip when writing data
to ‘output\_data’ ( integer )

\end{fulllineitems}

\index{build\_system() (in module build.cg\_build)@\spxentry{build\_system()}\spxextra{in module build.cg\_build}}

\begin{fulllineitems}
\phantomsection\label{\detokenize{build:build.cg_build.build_system}}\pysiglinewithargsret{\sphinxcode{\sphinxupquote{build.cg\_build.}}\sphinxbfcode{\sphinxupquote{build\_system}}}{\emph{cgmodel}}{}
Builds an OpenMM System() class object, given a CGModel() class object as input.

cgmodel: CGModel() class object

system: OpenMM System() class object

\end{fulllineitems}

\index{build\_topology() (in module build.cg\_build)@\spxentry{build\_topology()}\spxextra{in module build.cg\_build}}

\begin{fulllineitems}
\phantomsection\label{\detokenize{build:build.cg_build.build_topology}}\pysiglinewithargsret{\sphinxcode{\sphinxupquote{build.cg\_build.}}\sphinxbfcode{\sphinxupquote{build\_topology}}}{\emph{cgmodel}}{}
Construct an OpenMM topology for our coarse grained model

polymer\_length: Number of monomers in our coarse grained model
( integer )

backbone\_length: Number of backbone beads on individual monomers
in our coarse grained model, ( integer )

sidechain\_length: Number of sidechain beads on individual monomers
in our coarse grained model, ( integer )

\end{fulllineitems}



\chapter{OpenMM simulation tools for coarse grained modeling}
\label{\detokenize{simulation:openmm-simulation-tools-for-coarse-grained-modeling}}\label{\detokenize{simulation::doc}}
This page details the functions in cg\_openmm/src/simulation/.


\section{Replica exchange simulation tools for coarse grained modeling}
\label{\detokenize{simulation:replica-exchange-simulation-tools-for-coarse-grained-modeling}}

\section{General simulation tools for coarse grained modeling}
\label{\detokenize{simulation:module-simulation.tools}}\label{\detokenize{simulation:general-simulation-tools-for-coarse-grained-modeling}}\index{simulation.tools (module)@\spxentry{simulation.tools}\spxextra{module}}\index{get\_simulation\_time\_step() (in module simulation.tools)@\spxentry{get\_simulation\_time\_step()}\spxextra{in module simulation.tools}}

\begin{fulllineitems}
\phantomsection\label{\detokenize{simulation:simulation.tools.get_simulation_time_step}}\pysiglinewithargsret{\sphinxcode{\sphinxupquote{simulation.tools.}}\sphinxbfcode{\sphinxupquote{get\_simulation\_time\_step}}}{\emph{topology}, \emph{system}, \emph{positions}, \emph{temperature}, \emph{time\_step\_list}, \emph{total\_simulation\_time}}{}
Determine a valid simulation time step for our coarse grained model.

simulation: OpenMM simulation object

time\_step\_list: List of time steps for which to attempt a simulation in OpenMM.

time\_step: A time step that was successful for our simulation object.

\end{fulllineitems}

\index{minimize\_structure() (in module simulation.tools)@\spxentry{minimize\_structure()}\spxextra{in module simulation.tools}}

\begin{fulllineitems}
\phantomsection\label{\detokenize{simulation:simulation.tools.minimize_structure}}\pysiglinewithargsret{\sphinxcode{\sphinxupquote{simulation.tools.}}\sphinxbfcode{\sphinxupquote{minimize\_structure}}}{\emph{topology}, \emph{system}, \emph{positions}, \emph{temperature=Quantity(value=0.0}, \emph{unit=kelvin)}, \emph{simulation\_time\_step=None}, \emph{total\_simulation\_time=Quantity(value=1.0}, \emph{unit=picosecond)}, \emph{output\_pdb='minimum.pdb'}, \emph{output\_data='minimization.dat'}, \emph{print\_frequency=10}}{}
Construct an OpenMM simulation object for our coarse grained model.

topology: OpenMM topology object

system: OpenMM system object

positions: Array containing the positions of all beads
in the coarse grained model
( np.array( ‘num\_beads’ x 3 , ( float * simtk.unit.distance ) )

temperature: Simulation temperature ( float * simtk.unit.temperature )

simulation\_time\_step: Simulation integration time step
( float * simtk.unit.time )

output\_data: Name of output file where we will write the data from this
simulation ( string )

print\_frequency: Number of simulation steps to skip when writing data
to ‘output\_data’ ( integer )

\end{fulllineitems}



\chapter{Utilities for coarse grained modeling in OpenMM}
\label{\detokenize{utilities:utilities-for-coarse-grained-modeling-in-openmm}}\label{\detokenize{utilities::doc}}
This page details the functionality of utilities in cg\_openmm/src/utilities/util.py.

\phantomsection\label{\detokenize{utilities:module-utilities.util}}\index{utilities.util (module)@\spxentry{utilities.util}\spxextra{module}}\index{distance() (in module utilities.util)@\spxentry{distance()}\spxextra{in module utilities.util}}

\begin{fulllineitems}
\phantomsection\label{\detokenize{utilities:utilities.util.distance}}\pysiglinewithargsret{\sphinxcode{\sphinxupquote{utilities.util.}}\sphinxbfcode{\sphinxupquote{distance}}}{\emph{positions\_1}, \emph{positions\_2}}{}
Construct a matrix of the distances between all particles.

positions\_1: Positions for a particle
( np.array( length = 3 ) )

positions\_2: Positions for a particle
( np.array( length = 3 ) )

distance
( float * unit )

\end{fulllineitems}

\index{get\_box\_vectors() (in module utilities.util)@\spxentry{get\_box\_vectors()}\spxextra{in module utilities.util}}

\begin{fulllineitems}
\phantomsection\label{\detokenize{utilities:utilities.util.get_box_vectors}}\pysiglinewithargsret{\sphinxcode{\sphinxupquote{utilities.util.}}\sphinxbfcode{\sphinxupquote{get\_box\_vectors}}}{\emph{box\_size}}{}
Assign all side lengths for simulation box.

box\_size: Simulation box length ( float * simtk.unit.length )

\end{fulllineitems}

\index{set\_box\_vectors() (in module utilities.util)@\spxentry{set\_box\_vectors()}\spxextra{in module utilities.util}}

\begin{fulllineitems}
\phantomsection\label{\detokenize{utilities:utilities.util.set_box_vectors}}\pysiglinewithargsret{\sphinxcode{\sphinxupquote{utilities.util.}}\sphinxbfcode{\sphinxupquote{set\_box\_vectors}}}{\emph{system}, \emph{box\_size}}{}
Build a simulation box.

system: OpenMM system object

box\_size: Simulation box length ( float * simtk.unit.length )

\end{fulllineitems}



\chapter{Indices and tables}
\label{\detokenize{index:indices-and-tables}}\begin{itemize}
\item {} 
\DUrole{xref,std,std-ref}{genindex}

\item {} 
\DUrole{xref,std,std-ref}{modindex}

\item {} 
\DUrole{xref,std,std-ref}{search}

\end{itemize}


\renewcommand{\indexname}{Python Module Index}
\begin{sphinxtheindex}
\let\bigletter\sphinxstyleindexlettergroup
\bigletter{b}
\item\relax\sphinxstyleindexentry{build.cg\_build}\sphinxstyleindexpageref{build:\detokenize{module-build.cg_build}}
\indexspace
\bigletter{s}
\item\relax\sphinxstyleindexentry{simulation.tools}\sphinxstyleindexpageref{simulation:\detokenize{module-simulation.tools}}
\indexspace
\bigletter{u}
\item\relax\sphinxstyleindexentry{utilities.util}\sphinxstyleindexpageref{utilities:\detokenize{module-utilities.util}}
\end{sphinxtheindex}

\renewcommand{\indexname}{Index}
\printindex
\end{document}